% Define document class

\documentclass[8pt]{book}
\usepackage[utf8]{inputenc} 
\usepackage{niceframe}
\usepackage{swrule}
\usepackage{blindtext}
\usepackage{titlesec}
\usepackage{afterpage}
\usepackage{graphicx}
\usepackage[T1]{fontenc}
\usepackage{xcolor}
\usepackage{lmodern}
\usepackage{listings}
\usepackage[margin=1.4cm]{geometry}


\definecolor{mygreen}{rgb}{0,0.6,0}
\definecolor{mygray}{rgb}{0.5,0.5,0.5}
\definecolor{mymauve}{rgb}{0.58,0,0.82}


\lstset{
  basicstyle=\footnotesize\ttfamily,        % the size of the fonts that are used for the code
  breakatwhitespace=false,         % sets if automatic breaks should only happen at whitespace
  breaklines=false,                 % sets automatic line breaking
  captionpos=b,                    % sets the caption-position to bottom
  commentstyle=\color{mygreen},    % comment style
  extendedchars=true,              % lets you use non-ASCII characters; for 8-bits encodings only, does not work with UTF-8
  keepspaces=true,                 % keeps spaces in text, useful for keeping indentation of code (possibly needs columns=flexible)
  keywordstyle=\color{blue},       % keyword style
  language=[95]Fortran,            % the language of the code
  numbers=left,                    % where to put the line-numbers; possible values are (none, left, right)
  numbersep=5pt,                   % how far the line-numbers are from the code
  numberstyle=\tiny\color{mygray}, % the style that is used for the line-numbers
  rulecolor=\color{black},         % if not set, the frame-color may be changed on line-breaks within not-black text (e.g. comments (green here))
  showspaces=false,                % show spaces everywhere adding particular underscores; it overrides 'showstringspaces'
  showstringspaces=false,          % underline spaces within strings only
  showtabs=false,                  % show tabs within strings adding particular underscores
  stepnumber=1,                    % the step between two line-numbers. If it's 1, each line will be numbered
  stringstyle=\color{mymauve},     % string literal style
  tabsize=2,                       % sets default tabsize to 2 spaces
  title=\lstname                   % show the filename of files
}


\begin{document}


\chapter{Basics}


\section{Program Structure}
\lstinputlisting[caption={Basic program structure}]{source/program.f90}


\section{Data Types}
\lstinputlisting[caption={Using standard data types}]{source/data_types.f90}


\section{FORTRAN - Column Major}
\lstinputlisting[caption={Column major language}]{source/column_major.f90}





\chapter{Output}


\section{Print}
\lstinputlisting[caption={Print statement}]{source/output_print.f90}


\section{Write}
\lstinputlisting[caption={Write statement}]{source/output_write.f90}


\chapter{Arrays}

\section{Zero}
\lstinputlisting[caption={Zero (or any number)}]{source/arrays_zero.f90}

\section{Linspace}
\lstinputlisting[caption={Equally spaced linear space of points, as exists in Numpy}]{source/arrays_linspace.f90}








\chapter{Files}


\section{Write}
\lstinputlisting[caption={Writing to file}]{source/files_write.f90}




\section{Read to Array}
\lstinputlisting[caption={Read file into a 2D, 2 column, array}]{source/files_read_1.f90}


\section{Read Namelist}
\lstinputlisting[caption={Read namelist}]{source/files_read_namelist.f90}






\chapter{Interpolation}


\section{Lagrange}
\lstinputlisting[caption={Basic program structure}]{source/lagrange_polynomial_interpolation.f90}




\section{Lagrange - Larger Data Set}
\lstinputlisting[caption={Interpolate with k points over a larger data set}]{source/lp_interp_fill.f90}






\chapter{Differentiation}

\section{Zero}
\lstinputlisting[caption={Calculate 1st derivative}]{source/gradient1.f90}

\section{Linspace}
\lstinputlisting[caption={Calculate 2nd derivative}]{source/gradient2.f90}








\chapter{Integration}


\section{Simpson Integration Part 1}
\lstinputlisting[caption={Using Simpson Integration}]{source/simpson_integration.f90}


\section{Simpson Integration Part 2}
\lstinputlisting[caption={Using Simpson Integration}]{source/simpson_integration_2.f90}








\chapter{OpenMP}






\end{document}


